\chapter{Algorithmic Approaches}
\label{cha:algorithmic} % (labels for cross referencing)

\section{The Bernoulli Bandit Vs The Gaussian Bandit}
\label{sec:BernoulliBandit}

The Bernoulli bandit problem and the Gaussian bandit problem are both sequential decision-making problems:

In the Bernoulli bandit, the rewards are drawn from a Bernoulli distribution of unknown mean, which manifests as a binary result between two possible outcomes (normally success or failure). The goal is to select arms in a strategic manner to maximize the cumulative reward. This should balance exploration to learn more about other arm's expected means, and exploitation of the best-performing arms.

In contrast, the Gaussian bandit involves rewards drawn from a Gaussian distribution with unknown variance and mean, which manifests as a continuous outcome between [0, 1]. The objective is identical to the Bernoulli bandit, but the focus is on estimating unknown means and using information to make optimal decisions.

\subsection{Algorithmic Approaches}
\label{sec:Algorithms}

\subsubsection{Randomized}
\label{sec:randomized}
The randomized algorithm for multi-armed bandits involves selecting arms at random with equal probability, without considering any feedback or past performance. This algorithm is commonly referred to as the "purely random" or "uniform random" algorithm.

The random algorithm can be summarized as follows:

\pseudobox{%
  \KwIn{List of arms $\armsList$ with unknown distributions $\armDistributionVect$}
  \KwOut{Selection strategy}

  \BlankLine
  \For{$t = 1,2,\ldots$}{
    Choose a random arm $A_t$ uniformly from the set of arms $1,2,\ldots,K$;\newline
    Receive the reward $\reward{t}{}$ by pulling arm $A_t$;
  }
}{Randomized Algorithm}

At each time step $t$, the randomized algorithm randomly selects an arm $A_t$ from K arms, following a uniform distribution:

$$\text{For } i \in \{1, 2, \ldots, K\}: \Prob(A_t = i) = \frac{1}{K}$$

This algorithm is quite often used as a baseline for more sophisticated algorithm evaluation. In the long run, the average reward converges towards the discrepancy between the maximum and the average reward. More precisely, by the strong law of large numbers the following holds almost surely,

$$\lim_{{t \to \infty}} \frac{1}{t}\cdot \cumulativeRegret{t}{\policy} = \maxPopulationMean-\frac{1}{\numArms} \sum_{{k=1}}^{\numArms} \armPopulationMean{k}.$$

Although this algorithm is very simple to implement, it doesn't attempt to optimize arm selection based on past events. It can be described as a purely exploratory algorithm that doesn't exhibit any exploitative behavior, hence its performance is generally vastly inferior to that of any other algorithms.

Moreover, the value of \actionValueEstimate \space can significantly impact the algorithm's performance - in the extreme case where \actionValueEstimate=0, the algorithm will keep pulling the first arm that succeeds,preventing any further exploration.
In a practical sense, when implementing this in code, it could explore other arms when the programming language registers the success rate as so low, it rounds down to 0.
For example, Python will do this only when the success rate goes below $10^{-324}\%$ \cite{python_min_float}.

\subsubsection{Greedy}
\label{sec:Greedy}
The greedy algorithm is a step-up from randomly picking arms, since it actually takes into account past performance. Put simply, it chooses the arm that has the highest estimated expected reward, denoted by the arm with the highest current rate of success.

To begin, it assigns each arm with a "action-value estimate" \actionValueEstimate, essentially the predicted success rate of each arm. This is often set arbitrarily at some value, usually 0.5 or 1.

Then, at each time step $t$, the greedy algorithm selects the arm $A_t$ with the highest expected success rate.

The greedy algorithm can be summarized as follows:

\pseudobox{%
    \KwIn{List of arms $\armsList$ with unknown distributions $\armDistributionVect$, action-value estimate \actionValueEstimate}
    \KwOut{Selection strategy}
    \BlankLine
    \ForEach{arm $i = 1$ \KwTo $K$}{
        Initialise success rate $Q_i \leftarrow $\actionValueEstimate\newline
        Initialise number of successes $S_i \leftarrow \mathcal{0}$\newline
        Initialise count $C_i \leftarrow 0$
    }
    \BlankLine
    \For{$t = 1,2,\ldots$}{
        Choose arm $A_t = a$ with the highest success rate: $a \leftarrow \arg\max_i Q_i$\newline
        Receive the reward $R_t$ by pulling arm $a$;\newline
        Increment count: $N_a \leftarrow N_a + 1$\newline
        Increment successes: $S_a \leftarrow S_a + R_t$\newline
        Update success rate: $Q_a \leftarrow \frac{S_a}{N_a}$

  }
}{Greedy Algorithm}

Although this algorithm is simple and fast, it tends to exploit the immediate best option, and very rarely explores other options to try and optimise it's rewards.

However, the Greedy algorithm can exhibit linear regret under general conditions: : consider a 2-arm bandit with arms 0.9 and 0.8. Employing the Greedy algorithm may lead to linear regret, if it gets unlucky and gets a miss on arm 1, and a hit on arm 2, with probability 8\%. This essentially lock the algorithm into choosing this sub-optimal arm, and as a result, it could miss out on the higher reward arm, leading to persistent linear regret.

\subsubsection{Epsilon Greedy}
\label{sec:EpsilonGreedy}
The Epsilon Greedy algorithm is a combination of the Greedy and Randomized algorithm, addressing their tendencies to over-exploit and over-explore. It introduces a trade-off by incorporating an additional parameter $\epsilon$.

The core of the Epsilon Greedy algorithm is the same action-value estimate for each arm, as described in the Greedy algorithm. However, instead of always selecting the arm with the highest estimated success rate, the Epsilon Greedy algorithm randomly explores other arms with the hope of finding potentially better options. This choice is determined by sampling a Bernoulli distribution based on a probability of doing something random \epsilonFunction, which varies as a function of time $t$. Usually, this is either constant, linearly decreasing, or constant with a drop to 0.


The algorithm can be summarized as follows:

\pseudobox{%
    \KwIn{List of arms $\armsList$ with unknown distributions $\armDistributionVect$, action-value estimate \actionValueEstimate, sequence of exploration rates $(\epsilon_t)_{t \in [\timeHorizon]}$}
    \KwOut{Selection strategy}
    \BlankLine
    \ForEach{arm $i = 1$ \KwTo $K$}{
        Initialise success rate $Q_i \leftarrow $\actionValueEstimate\newline
        Initialise number of successes $S_i \leftarrow \mathcal{0}$\newline
        Initialise count $C_i \leftarrow 0$
    }
    \BlankLine
    \For{$t = 1,2,\ldots$}{
        \If{\epsilonFunction}{
            Choose a random arm $A_t \leftarrow$ Randomized(K)
        }
        \Else{
            Choose arm $A_t = a$ with the highest success rate: $a \leftarrow \arg\max_i Q_i$
        }
        Receive the reward $R_t$ by pulling arm $A_t$\newline
        Increment count: $N_{A_t} \leftarrow N_{A_t} + 1$\newline
        Increment successes: $S_{A_t} \leftarrow S_{A_t} + R_t$\newline
        Update success rate: $Q_{A_t} \leftarrow \frac{S_{A_t}}{N_{A_t}}$

}
}{Epsilon Greedy Algorithm}

Small \epsilonFunction values results in a more greedy strategy where the algorithm primarily exploits the current best arm. Conversely, a larger  \epsilonFunction values increases exploration and encourages the algorithm to try different arms more frequently.

While the Epsilon Greedy algorithm can mitigate the issues of the Greedy and Randomized algorithm, it is not without its shortcomings - set the exploration rates in \epsilonFunction too low, and the algorithm may get stuck in local sub-optimal arms and miss out on potentially better arms. On the other hand, too high values of \epsilonFunction means it might spend too much time exploring and not enough time exploiting the best arms, leading to, again, sub-optimal performance.

The Epsilon-Greedy algorithm can also exhibit linear regret under general conditions: consider a 2-arm bandit with arms 0.9 and 0.4. Using the Epsilon-Greedy algorithm with a fixed exploration rate \epsilonFunction = 0.5 $\forall t$. This initial randomness causes persistent reliable frequent sub-optimal selections, even as the algorithm learns and favors the better arm. The initial exploration-induced regret persists, resulting in an overall linear regret growth no less than (0.9 - (0.9+0.4)/2) = 0.25 per run. A good rule of thumb to follow is to never, in the long run, have \epsilonFunction $> \frac{1}{\numArms} - \frac{1}{\numArms^2}$, since this will cause the algorithm to, at worst, behave purely greedily


\subsubsection{Upper Confidence Bound (UCB)}
\label{sec:UCB}

\hrnote{https://www.jmlr.org/papers/volume3/auer02a/auer02a.pdf}

The Upper Confidence Bound algorithm (UCB) is a popular MAB algorithm that cleverly modifies the exploration rate using the confidence bounds around each arm. These help UCB estimate how likely an arm is to succeed, while factoring in how uncertain we are about this estimate.

The key idea is to choose arms that might give high rewards, while considering the uncertainty around their success rates. At each step, UCB selects the arm that strikes the best balance between exploiting its success rate and exploring its confidence bound. \ognote{Add "whisker plot" graph to visualise this}

In general terms, any UCB algorithm can be described by:

$$
A_t = \arg\max_i \left[ \text{exploitation term} + \text{exploration term} \right]
$$

where the exploration term tends to zero as $t \rightarrow \infty$. More specifically in this project, we will use this\cite{Roberts_2021} method\hrnote{add in other references}:

$$
A_t = \arg\max_i \left( Q_i + c \times \sqrt{\frac{\log(t)}{N_i}} \right)
$$

For some confidence value $c$. We use $c = 2$ for optimal regret. This allows us to derive the sensible confidence bounds, since it gives a probabilistic guarantee of how likely our sample mean is to deviate from the true mean of an arm distribution.

For UCB, finite sample concentration inequality-based (FSCI) confidence intervals are preferred over Central Limit Theorem (CLT) type confidence intervals due to their better performance when handling small sample sizes. CLT-based intervals rely on the assumption of having a large sample size (which ensures the sample mean converges to a normal distribution). However, in MAB problems, we sometimes have very limited data for some/all arms.

In contrast, FSCI-based confidence intervals such as Hoeffding inequality, give bounds on the deviation of the sample mean from the true mean with high probability, even with very small sample sizes. This helps account for the inherent randomness of MAB, and the exploration-exploitation trade-off, as arms that have a poor sample mean calculated from very few attempts still have a good-enough guess for their true mean.

The algorithm can be summarized as follows:

\pseudobox{%
    \KwIn{List of arms $\armsList$ with unknown distributions $\armDistributionVect$, confidence value $c$, sequence of time steps $(t)_{t \in [\timeHorizon]}$}
    \KwOut{Selection strategy}
    \BlankLine
    \ForEach{arm $i = 1$ \KwTo $K$}{
        Initialise success rate $Q_i \leftarrow $\actionValueEstimate\newline
        Initialise count $N_i \leftarrow 0$
    }
    \BlankLine
    \For{$t = 1,2,\ldots$}{
        Choose arm $A_t \leftarrow \arg\max_i \left( Q_i + c \sqrt{\frac{\log(t)}{N_i}} \right)$\;
        Receive the reward $R_t$ by pulling arm $A_t$\;
        Increment count: $N_{A_t} \leftarrow N_{A_t} + 1$\;
        Update success rate: $Q_{A_t} \leftarrow \frac{S_{A_t}}{N_{A_t}}$
    }
}{Upper Confidence Bound (UCB) Algorithm}

In the UCB algorithm, it aims to balance the exploitation of arms with high estimated success rates and the exploration of arms with high uncertainty or limited exploration history:

\begin{itemize}
    \item $Q_i$ chooses greedily which arm to pick given their past success rates
    \item $c \sqrt{\frac{\log(t)}{N_i}}$ attempts to sway the greedy exploration, by adding some uncertainty to the result.
    \begin{itemize}
        \item If an arm hasn't been pulled very often, the uncertainty is very large, and vice versa. Therefore, over time, the algorithm becomes more and more `confident' about it's estimates
        \item Due to the logarithmic nature of this particular uncertainty term, it'll slowly increase when arms aren't selected, but quickly shrink when they are. Therefore, the algorithm will tend to not `give up' on arms that appear to be sub-optimal, and will pull them infrequently until a significant score is obtained by another arm
    \end{itemize}
\end{itemize}

One of the key advantages of UCB is that it naturally adapts its exploration strategy over time. As the algorithm collects more data and gains confidence in the success rates of different arms, the exploration bonus diminishes, leading to a more exploitation-focused strategy. In comparison to the previous algorithms, UCB offers a more systematic approach to exploration by considering uncertainty explicitly. Depending on the problem at hand, UCB can provide a more efficient trade-off between exploration and exploitation, potentially leading to better overall performance in various applications.

\hrnote{https://tor-lattimore.com/downloads/book/book.pdf}

The regret bound for the UCB $\policy$ is sub-linear. Indeed, the rate is $O(\sqrt{n})$. \hrnote{possibly mention optimally.}

Finally, given a failure probability $\delta \in (0,1)$, we let $\ucb{a}{t}{\delta}$ denote the upper confidence bound on $\armPopulationMean{a}$ based on the first $t$ rounds,


\begin{theorem}[UCB regret bound]\label{thm:ucbRegretBound}
Suppose we have a stochastic bandit problem with time horizon $\timeHorizon \in \N$ and $\numArms \in \N$ with corresponding gaps $(\gap{i})_{i \in [\numArms]}$. Then letting $\ucbPolicy$ be the UCB policy with parameters $\delta$, $\totalFunction{a}{t}$ and $\empiricalMeanReward{a}{t}$, we have the following regret bound,
\begin{align*}
\cumulativeRegret{T}{\ucbPolicy} \leq 8 \sqrt{\timeHorizon \numArms \log(\timeHorizon)} + 3 \sum_{i=1}^{\numArms} \gap{i}.
\end{align*}
\hrnote{Return to: Given general formulation that yields gap dependent and gap independent bounds..}

\textcolor{red}{
For $\epsilon \in [0,1]$,
\begin{align*}
\cumulativeRegret{T}{\ucbPolicy} \leq n\epsilon + \sum_{i=1}^{\numArms}\one_{\{\Delta_i \geq \epsilon\}}\left(3 \gap{i}+\frac{16 \log n}{\Delta_i}\right).
\end{align*}
In particular, letting $\epsilon = \sqrt{16k \log n/n}$ we obtain 
\begin{align*}
\cumulativeRegret{T}{\ucbPolicy} \leq 8 \sqrt{\timeHorizon \numArms \log(\timeHorizon)} + 3.
\end{align*}
($n$ should be $\timeHorizon$)
}

\end{theorem}



We shall now give a high-level description of the proof. For completeness we give the full proof 

We proceed as follows. Recall that $\totalFunction{i}{\timeHorizon}$ denotes the number of times arm $i \in [\numArms]$ has been played over the time horizon $\timeHorizon$. We break down the regret for the UCB policy,
$$\cumulativeRegret{T}{\ucbPolicy} \leq \sum_{i=1}^{\numArms} \gap{i} \cdot \Ex(\totalFunction{i}{\timeHorizon}).$$

Next, using the Law of Total Expectation, we can split up the $\Ex(\totalFunction{i}{t})$ by using a ``good event'', which is when:

\begin{enumerate}
    \item The best arm's mean is never underestimated
    \item The other's arm's upper bounds are never higher than the best arm's mean
\end{enumerate}

So, we get:

\ognote{Maybe don't need the middle part here since it's a high-level overview}

$$\Ex(\totalFunction{i}{t}) = \Ex(\mathbb{I}(G_i)\totalFunction{i}{t}) + \Ex(\mathbb{I}(G_i^c)\totalFunction{i}{t}) \leq \armPopulationMean{i} + \Prob(G_i^c) \cdot n.$$

We apply Hoeffding's inequality (\hrnote{ref}) obtain the bound,
$$\Prob(G_i^c) \leq n \delta + \exp\left(-\frac{ \gap{i}^2}{4} \left\lceil \frac{8\log(1/\delta)}{\gap{i}^2} \right\rceil\right)\leq (n+1)\delta.$$
\hrnote{replace the above bound with one only involving $\Delta_i$, $\numArms$, $\delta$.}

\hrnote{consider removing from here to ..}For some $u_i$ and $c$. The full proofs outlines why $u_i = \lceil \frac{2log(1/\delta)}{(1-c)^2\gap[i]^2} \rceil$ is ideal. \hrnote{.. here..} After setting $\delta = 1/n^2$ \ognote{Do I need to say why here?}, this leads to:

$$\Ex(\totalFunction{i}{t}) \leq 3 + \frac{16log(t)}{\gap{i}^2}$$

Which neatly resolves when plugged back into the original equation to give:

$$R_t(\policy) \leq 8 \sqrt{tk \log(t)} + 3 \sum_{i=1}^{\numArms} \gap{i}$$


\subsubsection{(Bernoulli) Thompson Sampling}
\label{sec:BernoulliThompsonSampling}
Thompson Sampling is another popular approach to MAB - unlike the Upper Confidence Bound (UCB) algorithm, Thompson Sampling approaches the problem from a Bayesian perspective, leveraging posterior probability distributions to make decisions about which arms to pull.

At its core, Thompson Sampling maintains a probabilistic model of the true success rate distribution for each arm. Instead of trying to estimate a single success rate for each arm with a single value, Thompson Sampling maintains an entire posterior distribution of success rates, which is updated as the algorithm collects more data.

The algorithm is named after William R. Thompson, who introduced it in his paper ``On the Likelihood that One Unknown Probability Exceeds Another in View of the Evidence of Two Samples'' \cite{Thompson_1933} in 1933. The central idea behind Thompson Sampling is to select arms for exploration and exploitation according to their sampled success rates from their respective distributions.

The Thompson Sampling algorithm can be summarized as follows:

\pseudobox{%
    \KwIn{List of arms $\armsList$ with unknown distributions $\armDistributionVect$, sequence of time steps $(t)_{t \in [\timeHorizon]}$}
    \KwOut{Selection strategy}
    \BlankLine
    \For{$t = 1,2,\ldots$}{
        Sample a success rate $\theta_i$ from the distribution $\armDistribution{i}$ associated with arm $i$ for all $i \in \numArms$\;
        Choose arm $A_t \leftarrow \arg\max_i \theta_i$\;
        Receive the reward $R_t$ by pulling arm $A_t$\;
        Update the distribution parameters for arm $A_t$ based on the observed reward;
    }
}{Thompson Sampling Algorithm}

Since we are discussing the Bernoulli bandit, the appropriate algorithm is the Bernoulli Thompson Sampling, often abbreviated to BernTS, which is outlined below. It utilises the Beta distribution, which is very suitable for modeling probabilities in Bernoulli trials.

Importantly, this utilizes a "conjugate prior", which describes a prior distribution that, when combined with the likely-hood function, gives a posterior distribution in the same family as the prior. Specifically in the case , the Beta distributions acts as the Bernoulli's conjugate prior:

$$Beta(\alpha,\beta) \times Bernoulli(p) \propto Beta(\alpha',\beta')$$

This property greatly simplifies the process of updating beliefs based on incoming evidence, since it's much easier to calculate compared to e.g Dirichlet (used for vector uncertainty), where $Dirichlet(\alpha_1,\alpha_2) \times Bionomial \not\propto Dirichlet(\alpha_1',\alpha_2')$

Additionally, each arm $i \in [0, \numArms]$ is associated with a Beta distribution:

$$\armDistribution{i} \sim \mathrm{Beta}(S_i + \alpha_i, F_i + \beta_i)$$

In this algorithm, priors are introduced - any initial beliefs or assumptions about the environment before any data is collected. $\alpha$ represents past "positive" experience, and conversely $\alpha$ represents past "negative" experience. By incorporating priors into BernTS, it starts with a base understanding of each arm's probabilities, which is refined over time, slowly becoming less influential as arms are explored, aiding the exploration/exploitation trade-off.

\pseudobox{%
    \KwIn{List of arms $\armsList$ with Bernoulli success probabilities $\theta_i$, sequence of time steps $(t)_{t \in [\timeHorizon]}$, priors $\alpha_i, \beta_i$}
    \KwOut{Selection strategy}
    \BlankLine
    \ForEach{arm $i = 1$ \KwTo $K$}{
        Initialise number of successes $S_i \leftarrow 0$\;
        Initialise number of failures $F_i \leftarrow 0$\;
    }
    \For{$t = 1,2,\ldots$}{
        Sample a success probability $\theta_i$ from $\text{Beta}(S_i + \alpha_i, F_i + \beta_i)$ associated with arm $i$ for all $i \in \numArms$\;
        Choose arm $A_t \leftarrow \arg\max_i \theta_i$\;
        Receive the reward $R_t$ by pulling arm $A_t$\;
        \If{$R_t = 1$}{
            $S_i \mathrel{+}= 1$
        }
        \Else{
            $F_i \mathrel{+}= 1$
        }
    }
}{Bernoulli Thompson Sampling Algorithm}


Thompson Sampling finds a balance between exploration and exploitation, since it has a built-in exploration mechanism, occasionally sampling arms that may not be currently considered the best based on existing data. Over time, the algorithm's estimates improve as more data is collected, leading to more accurate decisions.

Thompson Sampling's key advantages are its simplicity and adaptability. It naturally handles uncertainty and adjusts its exploration strategy based on observed data. This adaptability makes it particularly effective when arm success rate distributions may change over time.

However, it struggles to manage the computational complexity of sampling from complex distributions, especially when dealing with a large number of arms. Also, like other bandit algorithms, Thompson Sampling's performance can be affected by the choice of hyper-parameters (priors). Incorrect choices here can cause the algorithm to produce very unreliable results.

\ognote{Probably should add a graphed example of TS returning complete rubbish with poor hyper-parameters, just to make the point}


\subsubsection{Ripple Sampling}
\label{sec:ripple}

\ognote{Some sort of intro and explaination should go here}

The algorithm can be summarized as follows:

\pseudobox{%
    \KwIn{List of arms $\armsList$ with unknown distributions $\armDistributionVect$, pessimism tolerance $\rho$, numerical tolerance $\epsilon$, sequence of time steps $(t)_{t \in [\timeHorizon]}$}
    \KwOut{Selection strategy}
    \BlankLine
    \ForEach{arm $i = 1$ \KwTo $K$}{
        Initialise success count $S_i \leftarrow 0$\newline
        Initialise number of plays $M_i \leftarrow 0$
    }

    \BlankLine
    \SetKwFunction{normProb}{normProb}
    \SetKwProg{Fn}{Function}{:}{}
    \Fn{\normProb{$x, s, m$}}{
        $a \leftarrow \frac{s}{m}$\;
        \Return $\frac{x^{s} (1-x)^{m-s}}{a^{s}  (1-a)^{m-s}}$\;
    }
    
    \BlankLine
    \For{$t = 1,2,\ldots$}{
        \For{$i \in [K]$}{
            Use binary search to find $X_i$ so $|\mathrm{normProb}(X_i, S_i, M_i) - \rho| \leq \epsilon$\;
            }
        Choose arm $A_t \leftarrow \arg\max_i X_i$\;
        Receive the reward $R_t$ by pulling arm $A_t$\;
        $M_i = M_i + 1$\;
        \If{$R_t = 1$}{
            $S_i \leftarrow S_i + 1$\;
        }
    }
}{Ripple Algorithm}

\ognote{Would be interesting to include some sort of lower bound proof here}

\hrnote{Connections to Thompson sampling, and Malliard sampling}